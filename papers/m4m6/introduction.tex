\section{Introduction}
\label{sec:introduction}

In this paper, we describe \oops\/, an {\it O}bject {\it O}riented {\it P}rover for $\mathbf\mathit{S}\textbf{5}_{(n)}$. It is a tableaux-based theorem prover, specifically designed for use in a classroom setting. Although not as efficient as highly optimized provers like MSPASS \citep{mspass2000} and FaCT \citep{fact1998}, or as flexible as highly extensible provers like LoTReC \citep{lotrec2005} and the Tableau Workbench \citep{twb2009}, we do think \oops\ offers a unique combination of features that make it particularly suitable for educational purposes.

First, \oops\ offers support for $\mathbf\mathit{S}\textbf{5}_{(n)}$, the logic typically used to model human reasoning processes. As far as we are aware, there are currently no other theorem provers which do this. Furthermore, \oops\ is capable of visualizing its tableaux proofs, and of generating counterexamples to formulas that are false. These two properties are very useful for educational proof tools, and they are shared by at least two other systems: LoTReC \citep{lotrec2005} can display its tableaux trees, and the Logics Workbench \citep{heuerding1996} can print its derivations in sequent calculi.

Another feature \oops\ has in common with LoTReC \citep{lotrec2005}, as well as MSPASS \citep{mspass2000}, but with few other theorem provers, is its free and convenient distribution. \oops\ is packaged as a ZIP file that includes all dependencies, and once extracted, can be run simply by double-clicking the resulting oops.jar file. This is true for any operating system, provided the Java VM is available. This in contrast to the Tableau Workbench \citep{twb2009}, which must be built from source, FaCT \citep{fact1998}, which offers binaries only for Linux and Windows, and otherwise requires Lisp, and the Logics Workbench \citep{heuerding1996}, which is not open source, and can be difficult or impossible to install, due to outdated dependencies.

A fifth property that makes \oops\ particularly attractive for educational use is its graphic user interface. Although simple, its GUI allows students to input formulas in an easy and intuitive format, or to save and load files. LoTReC \citep{lotrec2005}, MSPASS \citep{mspass2000} and the the Logics Workbench \citep{heuerding1996} also offer this functionality.

 \oops\/'s final educational attribute is its integrated scripting facility. In our experience, most student projects involving theorem provers are efforts to model riddles, or game situations. This requires that students be able to build and extend theories, using loops and conditionals where necessary. Many proof tools, like the Logics Workbench, offer custom scripting languages for this purpose. \oops\/, by contrast, integrates the existing Lua\footnote{http://www.lua.org/} scripting language. Lua can be used to call most of \oops\/'s functions, and offers an  rich input language.
 
In the rest of this paper, we first offer a technical description of \oops\, including its proof system, its formal properties, its implementation in Java, and the details of its input language. This is followed by a more detailed description of \oops\/'s educational properties, as well as a worked out example. We conclude with a discussion of limitations and future work.