\documentclass[11pt,a4paper]{article}
\usepackage[english]{babel}
\usepackage{latexsym}
\usepackage{amssymb,amsmath}
\usepackage{theorem}
\usepackage{algorithm2e}
\newcommand{\fn}[1]{{\sc #1}}
\theorembodyfont{\upshape}
\newtheorem{defs}{Definition}[section]
\newtheorem{exs}{Example}[section]


\author{Elske van der Vaart \and Gert van Valkenhoef}
\title{OOPS: An Automated Proof Tool for $\textbf{S5}_{(n)}$}

\begin{document}

\maketitle

\section{Introduction}

This paper describes {\tt OOPS}, an {\it O}bject {\it O}riented {\it P}rover for $\mathbf\mathit{S}\textbf{5}_{(n)}$. {\tt OOPS} is a Java-based implementation of semantic tableaux, which can automatically prove or disprove  formulas of epistemic logic, using the $\textbf{S5}_{(n)}$ axioms. It was developed as a final project for the course Multi-Agent Systems at the Rijksuniversiteit Groningen.
\par {\tt OOPS} is implemented as a Java package (code library) in order to facilitate reuse in a Logics Workbench-like ~\cite{wb} tool as well as an online applet. The choice for Java sets {\tt OOPS} apart from similar tools written in Prolog or C in terms of portability and deployment options.

\section{Semantic Tableaux for $\textbf{S5}_{(n)}$}
\label{theory}
{\tt OOPS} generates proofs using de Boer's ~\cite{these} proof system {\bf ELtap}, which is an extension of Fitting and Mendelsohn's ~\cite{fit_men} labelled semantic tableau method. Within this formalism, semantic tableau proofs are represented as trees of labelled formulas, where labels specify accessibility relations between worlds.

\begin{defs}{\it Labels} \newline
If $\sigma$ is a label representing a world $m$, then $\sigma.n_i$ represents a world $n$ accessible from $m$ by agent $i$, where $\sigma,m,n,i \in \mathcal{N}$. The {\it length} of a label $\sigma$ is the number of dots it contains plus one.~\cite{beck_gore} \end{defs} 

In this type of semantic tableau, the formula $\varphi$ to be proved is placed at the top of the tree in negated form, $\neg \varphi$, as the {\it root node}. Additional nodes are added to the tree by application of {\it tableau extension rules}. All {\it propositional extension rules} create child nodes with labels identical to those of the parent nodes.

\begin{defs} {\it Propositional Tableau Extension Rules} ~\cite{these, fit_men}
\label{proptab}
\newline
\begin{tabular}{cccc}

&&$\neg\neg$ = & \\ 
&&\begin{tabular} {lc}
$\sigma$&$\neg\neg\varphi$\\
\hline
$\sigma$&$\varphi$
\end{tabular} & 

\begin{tabular} {c}
{\it Double Negation}\\{\it Rule}
\end{tabular} \\  [14pt]

$\wedge_\wedge$ = & $\wedge_\vee$ = & $\wedge_\rightarrow$ = & \\

\begin{tabular} {lc}
$\sigma$&$\varphi\wedge\psi$\\
\hline
$\sigma$&$\varphi$\\
$\sigma$&$\psi$
\end{tabular} &

\begin{tabular} {lc}
$\sigma$&$\neg(\varphi\vee\psi)$\\
\hline
$\sigma$&$\neg\varphi$\\
$\sigma$&$\neg\psi$
\end{tabular} &

\begin{tabular} {lc}
$\sigma$&$\neg(\varphi\rightarrow\psi)$\\
\hline
$\sigma$&$\varphi$\\
$\sigma$&$\neg\psi$
\end{tabular} &

\begin{tabular} {c}
{\it Conjunctive}\\{\it Rules}
\end{tabular} \\ [24pt]

$\vee_\wedge$ = & $\vee_\vee$ = & $\vee_\rightarrow$ = & \\

\begin{tabular} {lrlr}
\multicolumn{4}{c}{$\sigma\;\;\;\;\neg(\varphi\wedge\psi)$}\\
\hline
$\sigma$&$\neg\psi$&\multicolumn{1}{|r}{$\sigma$}&$\neg\psi$
\end{tabular} &

\begin{tabular} {lrlr}
\multicolumn{4}{c}{$\sigma\;\;\;\;\varphi\vee\psi$}\\
\hline
$\sigma$&$\psi$&\multicolumn{1}{|r}{$\sigma$}&$\psi$
\end{tabular} &

\begin{tabular} {lrlr}
\multicolumn{4}{c}{$\sigma\;\;\;\;\varphi\rightarrow\psi$}\\
\hline
$\sigma$&$\neg\psi$&\multicolumn{1}{|r}{$\sigma$}&$\psi$
\end{tabular} & 

\begin{tabular} {c}
{\it Disjunctive}\\{\it Rules}
\end{tabular} \\ [18pt] \\
\end{tabular}
\end{defs}

A node without children is called a {\it leaf}, while a {\it branch} is any path from the root node to a leaf node. {\it Conjunctive rules} add two child nodes to the same branch, while {\it disjunctive rules} fork a branch in two, adding one child node to each. For a succesful tableau proof of a formula $\varphi$, all branches from the root node $\neg\varphi$ must be {\it closed}.

\begin{defs}{\it Closure of Branches} \newline
A tableaubranch is {\it closed} if it contains two nodes with the formulas $\psi$ and $\neg\psi$, where each has the same label $\sigma$. Otherwise, a branch is {\it open}.
\end{defs}

\begin{defs}{\it Closure of Tableaux} \newline
A semantic tableau is {\it closed} if all its branches are closed. A closed semantic tableau with root node $\neg\varphi$ is a {\it proof} of $\varphi$.
\end{defs}

Example 1.1 demonstrates how a semantic tableau can be used to prove a propositional formula. From left to right, each node is represented by its label $\sigma$, its formula $\psi$, an index $k$ and an abbreviated rule $R$. The indexes indicate the order in which nodes were added to tree, while the abbreviated rules summarize how they were generated. A $\otimes$ indicates a closed branch, while the $\dagger$ joins the two nodes that close it.

\begin{exs}{\it Tableau Proof of $(p \rightarrow q) \rightarrow (\neg p \vee q)$}

\begin{center}
\begin{tabular} {p{50pt}lp{135pt}rrp{50pt}}
& $\sigma$ & $\psi$ & $k$ & $R$ & \\
& 1 & $\neg((p \rightarrow q ) \rightarrow (\neg p \vee q ))$ & 1. &  & \\
& 1 & $p \rightarrow q$ & 2. & $\wedge_{1_\rightarrow}$ : 1 & \\
& 1 & $\neg(\neg p \vee q)$ & 3. & $\wedge_{2_\rightarrow}$ : 1 &\\
& 1 & $\neg\neg p$ & 4. & $\wedge_{1_\vee}$ :  3 & \\
& 1 & $\neg q$ & 5. & $\wedge_{2_\vee}$ :  3 & \\
\end{tabular} \newline
\begin{picture}(500,100)
\put(175,100){\line(3,-2){65}}
\put(175,100){\line(-3,-2){65}}
\put(80,25){\makebox(0,0){\begin{tabular} {lp{20pt}rr}
1 & $\neg p$ & 6. & $\vee_{1_\rightarrow}$ : 2 \\
1 & $p$ & 8. & $\neg\neg$  :  4 \\
& $\otimes$ & & (6$\dagger$8)
\end{tabular}}}
\put(265,25){\makebox(0,0){\begin{tabular} {lp{20pt}rr}
1 & $q$ & 7. & $\vee_{2_\rightarrow}$ : 2 \\ 
& $\otimes$ & & (5$\dagger$7) \\ &
\end{tabular} }}
\end{picture}

\end{center}
\end{exs}

For use with $\textbf{S5}_{(n)}$, the propositional expansion rules in Definition \ref{proptab} must be extended with modal expansion rules, as de Boer~\cite{these} does for his proof system {\bf ELtap}. {\it Possibility rules} introduce new labels to a branch, while {\it necessity rules} add formula's to existing labels. {\it Basic necessity rules} follow from $\textbf{K}$; {\it special necessity rules} are derived from the extra axioms of $\textbf{S5}_{(n)}$. If a node's label is of the form $\sigma.k_i$ (i.e., it represents a world $k$ accessible from $\sigma$ by agent $i$), and its formula $\varphi$ concerns the same agent $i$ (i.e., $\varphi = \Box_i\psi$ or $\varphi = \Diamond_i\psi$), only {\it starred $*$} extension rules may be applied. In all other cases, {\it regular} rules must be applied.

\begin{defs} {\it Modal Tableau Extension Rules} ~\cite{these, fit_men}
\label{modtab}
\begin{center}
\begin{tabular} {cccc}
& $M_\Diamond$ = & $M_\Box$ = & \\
 \begin{tabular} {c}
where $k_i = null$\\
or $i \neq j$ \end{tabular} &
 
\begin{tabular} {lc}
$\sigma.k_i$&$\Diamond_j\varphi$\\
\hline
$\sigma.k_i.n_j$&$\varphi$
\end{tabular} & 

\begin{tabular} {lc}
$\sigma.k_i$&$\neg \Box_i\varphi$\\
\hline
$\sigma.k_i.n_j$&$\neg\varphi$
\end{tabular} &

\begin{tabular} {c}
{\it Possibility}\\{\it Rules}
\end{tabular} \\ [18pt]

& $M_\Diamond*$ = & $M_\Box*$ = & \\

 \begin{tabular} {c}
where $k_i \neq null$
\end{tabular} &
 
\begin{tabular} {lc}
$\sigma.k_i$&$\Diamond_i\varphi$\\
\hline
$\sigma.n_i$&$\varphi$
\end{tabular} & 

\begin{tabular} {lc}
$\sigma.k_i$&$\neg \Box_i\varphi$\\
\hline
$\sigma.n_i$&$\neg\varphi$
\end{tabular} &

\begin{tabular} {c}
{\it Possibility} \\ {\it Rules$*$}
\end{tabular} \\ [18pt]

& $K_\Box$ = & $K_\Diamond$ = & \\

 \begin{tabular} {c}
where $k_i = null$ \\
or $i \neq j$
\end{tabular} &
 
\begin{tabular} {lc}
$\sigma.k_i$&$\Box_j\varphi$\\
\hline
$\sigma.h_j$&$\varphi$
\end{tabular} & 

\begin{tabular} {lc}
$\sigma.k_i$&$\neg \Diamond_j\varphi$\\
\hline
$\sigma.h_j$&$\neg\varphi$
\end{tabular} &

\begin{tabular} {c}
{\it Basic Necessity}\\{\it Rules}
\end{tabular} \\ [18pt]

& $K_\Box*$ = & $K_\Diamond*$ = & \\

 \begin{tabular} {c}
where $k_i \neq null$ 
\end{tabular} &
 
\begin{tabular} {lc}
$\sigma.k_i$&$\Box_i\varphi$\\
\hline
$\sigma.h_i$&$\varphi$
\end{tabular} & 

\begin{tabular} {lc}
$\sigma.k_i$&$\neg \Diamond_i\varphi$\\
\hline
$\sigma.h_i$&$\neg\varphi$
\end{tabular} &

\begin{tabular} {c}
{\it Basic Necessity} \\ {\it Rules$*$}
\end{tabular} \\ [18pt]

& $T_\Box$ = & $T_\Diamond$ = & \\

 \begin{tabular} {c}
where $k_i = null$ \\
or $i \neq j$
\end{tabular} &
 
\begin{tabular} {lc}
$\sigma.k_i$&$\Box_j\varphi$\\
\hline
$\sigma.k_i$&$\varphi$
\end{tabular} & 

\begin{tabular} {lc}
$\sigma.k_i$&$\neg\Diamond_j\varphi$\\
\hline
$\sigma.k_i$&$\neg\varphi$
\end{tabular} & 

\begin{tabular} {c}
{\it Special Necessity}\\{\it Rules}
\end{tabular} \\ [18pt]

& $R_\Box*$ = & $R_\Diamond*$ = & \\

 \begin{tabular} {c}
where $k_i \neq null$
\end{tabular} &

\begin{tabular} {lc}
$\sigma.k_i$&$\Box_i\varphi$\\
\hline
$\sigma$&$\varphi$
\end{tabular} &

\begin{tabular} {lc}
$\sigma.k_i$&$\neg\Diamond_i\varphi$\\
\hline
$\sigma$&$\neg\varphi$
\end{tabular} &

\begin{tabular} {c}
{\it Special Necessity}\\{\it Rules$*$}
\end{tabular} \\ [20pt]

\end{tabular}
\end{center}
\end{defs}

Intuitively, we can understand these rules as follows. Let's consider $\Box_i \varphi$ as an example. In a world $k$, a formula $\Box_i \varphi$ implies 'in all worlds accessible from $k$ by agent $i$, $\varphi$ holds'. Given the reflexive, transitive nature of $\textbf{S5}_{(n)}$, 'all worlds accessible from $k$' includes $k$ itself, all worlds $l$ accessible from $k$, all worlds $m$ from which $k$ is accessible, and all worlds $o$ accessible from $m$. This corresponds to rules $T$, $K_\Box$, $R*$ and $K_\Box*$, respectively.

As is evident from Definition \ref{modtab}, modal tableau extension rules add two complications to the propositional tableau scheme: Labels are now important, and multiple rules may apply to the same formula. We can summarize the possible applications of the different extension rules as follows, taking into account the fact that $\Diamond \equiv \neg\Box\neg$:

\paragraph \noindent In a node with a label of the form $\sigma.k_i$ and a formula $\varphi$, where $\varphi = \Box_i\psi$ or $\varphi = \Diamond_i\psi$:
\begin{itemize}
\item A {\large  $\Diamond_i\varphi$} can only be expanded into a node with label $\sigma.n_i$ (rule $M_\Diamond*$), with formula $\varphi$.
\item A {\large $\Box_i \varphi$} can be expanded into a node with label $\sigma$ (rule $R*$), or a node with label $\sigma.h_i$ (rule $K_\Box*$), both with formula $\varphi$.
\end {itemize}

In a node with a label of the form $\sigma$, i.e. a label of length == 1, {\it or} a label of the form $\sigma.k_i$ and a formula $\varphi$, where $\varphi = \Box_j\psi$ or $\varphi = \Diamond_j\psi$ and $i \neq j$:
\begin{itemize}
\item A {\large  $\Diamond_i\varphi$} can be expanded only into a node with label $\sigma.n_i$ (rule $M_\Diamond$), with formula $\varphi$.
\item A {\large $\Box_i \varphi$} can be expanded into a node with label $\sigma$ (rule $T$) or a node with label $\sigma.h_i$ (rule $K_\Box$), both with formula $\varphi$.
\end {itemize}

\begin{exs}{\it Tableau Proof of $\Diamond_1(p \wedge \Box_1 q) \rightarrow (\Diamond_1 p\wedge \Box_1 q)$}
\begin{center}
\begin{tabular} {p{40pt}lp{155pt}rrp{40pt}}
& $\sigma$ & $\psi$ & $k$ & $R$ & \\
& 1 & $\neg(\Diamond_1(p \wedge \Box_1 q) \rightarrow (\Diamond_1 p\wedge \Box_1 q))$& 1. &  & \\
& 1 & $\neg\Diamond_1(p \wedge \Box_1 q) $ & 2. & $\wedge_{1_\rightarrow}$ : 1 & \\
& 1 & $\Diamond_1 p\wedge \Box_1 q$ & 3. & $\wedge_{2_\rightarrow}$ : 1 &\\
& 1 & $\Diamond_1 p$ & 4. & $\wedge_{1_\vee}$ :  3 & \\
& 1 & $\Box_1 q$ & 5. & $\wedge_{2_\vee}$ :  3 & \\
& 1.1 & $p$ & 6. & $M\Diamond$ : 4 & \\
& 1.1 & $\neg(p \wedge \Box_1 q)$ & 7. & $K\Diamond$ :  2 & \\
\end{tabular} \newline
\begin{picture}(500,100)
\put(185,110){\line(3,-2){65}}
\put(185,110){\line(-3,-2){65}}
\put(90,25){\makebox(0,0){\begin{tabular} {lp{20pt}rr}
1.1 & $\neg p$ & 8. & $\vee_{1_\wedge}$ : 7 \\
& $\otimes$ & & (6$\dagger$8) \\ & \\ &
\end{tabular}}}
\put(275,25){\makebox(0,0){\begin{tabular} {lp{20pt}rr}
1.1 & $\neg\Box_1 q$ & 9. & $\vee_{2_\wedge}$ : 7 \\ 
1.2 & $\neg q$ & 10. & $M_\Box*$ : 9 \\ 
1.1 & $q$ & 11. & $K_\Box*$ : 10 \\ 
& $\otimes$ & & (10$\dagger$11) \\ 
\end{tabular} }}
\end{picture}

\end{center}
\end{exs}

\section{Use of {\tt OOPS}}
\label{use}

\begin{tabular}{cc}
Logical Operator & OOPS Input Symbol \\
$\neg$ & \tt{$\sim$} \\ 
$\wedge$ & \tt{\&} \\
$\vee$ & \tt{|} \\
$\rightarrow$ & \tt{>} \\
$\leftrightarrow$ & \tt{=} \\
$\Box_i$ & \tt{\#\_i}  \\
$\Diamond_i$ & \tt{\%\_i}  \\
\end{tabular}

\section{Implementation of {\tt OOPS}}
\label{imp}
The {\tt OOPS} package makes extensive use of the new `generics' feature in Java 1.5 in
order to ensure readability and type safety. Most of the code is covered by
Javadoc-style comments that enable automatic generation of API documentation,
which is also available online.

A JUnit 4 testing framework is in place, providing partial testing coverage
for the library. Whenever bugs were encountered, a unit test was added to the
test suite to prevent them from reappearing.

Additionally, several critical parts of the package are covered by an array of
basic tests (although more elaborate tests are desirable) and a set of 25
test formulas (both positive and negative) is available. Tests can be run
automatically by the build tool.

A SableCC based parser for multi modal formulas is included in the package and
a command-line tool that can prove multi modal formulas is provided for
testing and reference purposes. 

Formulas are represented by a collection of objects that implement the {\tt
Formula} interface. Each formula implements functions that check for equality,
perform matching and perform substitution for variables. A formula $\varphi$
is defined to match another formula $\psi$ if a substitution $S$ for the free
variables in $\varphi$ can be found such that $\left[ S \right] \varphi =
\psi$. Formulas and agents can both be variables within a formula, so matching
will return a substitution of formulas for variables and a substitution of
agents for variables.

Labels are defined recursively as a structure containing a world, an agent and
a parent label. All labels implement the {\tt Label} interface. The recursion
is terminated by a {\tt NullLabel} parent label. For any world, the top label
is represented as {\tt NullLabel.Label(world0, NullAgent)}. By definition, no
agent is equal to the {\tt NullAgent}. Using this representation of the top
world, the $k_i = null$ (and similar) conditions for rules do not need to be
implemented explicitly, but follow automatically from the form of the top
label. Labels also have methods to check for equality, perform matching
(returning a substitution for variables) and substitution.

A {\tt Node} in a semantic tableau is a combination of a Label and a Formula.
Nodes can be checked for equality, matched and variables can be substituted
for.

Rules are represented by objects implementing the {\tt Rule} interface. Rules
can be matched to Nodes and contain rewrite rules that result in new Nodes.
There are two types of propositional rules (linear and split) and two types of
modal rules (create - introducing new worlds and access - introducing new
formulas that are necessarily true in a certain set of worlds).

The tableau operates on a queue of matches. On every iteration a new match is
taken from the front of the queue and handled based on the type of rule it
resulted from. For a linear rule, the resultant nodes are placed on the branch
and the matches that can be created from these nodes are put in the queue. For
a split rule, a new tableau is created for each of the resultant nodes. Create
rules create a new world (if this hasn't been attempted yet) and known
necessities are applied to this new world. Access rules introduce new
necessities which are applied to all existing worlds immediately.

A tableau closes if all branches (created by split rules) close. A branch
closes if it closes for any label on the branch.

Infinite recursion is avoided by storing matches from the create rules and
ignoring any duplicates. Non-determinism and backtracking are avoided by
eventually applying all possible rules (when a node matches more that one
rule, all possibilities are explored). The implications for computational
complexity have not been explored in-depth.

See appendix \ref{app:propalg} for an overview of a suitable tableau algorithm
for propositional formulas (this simple algorithm was implemented as a
prototype for the modal algorithm, which replaces it). The modal algorithm is
listed in appendix \ref{app:modalg}.

The queue used by the tableau is a priority queue, in which rules that are
less `bad' are given higher priority. We hope that this heuristic will prevent
the unnecessary introduction of branches and worlds in many cases, thus
closing the tableau as early as possible. The optimal assignment of badness
values to rules is not known (although a good approximation has been
implemented).


\section{Validation of {\tt OOPS}}

Although full validation of {\tt OOPS} was outside the scope of this project, it was succefully tested on the set of example formulas listed by de Boer ~\cite{these} (see Appendix A.1.4 of that work). This example set  consists of 17 formulas that are true in $\textbf{S5}_{(n)}$, as well as 8 that are not. {\tt OOPS} performs correctly on all 25. This includes modal formulas concerning up to three different agents, with up to six nested modal operators. For full details of the example set, we refer to de Boer's ~\cite{these} thesis, or our own implementing code. On the 2.66 GHz Intel Pentium Dual-Core Processor we used to test {\tt OOPS}, evaluating all 25 examples takes slightly less than half a second.

\label{val}

\section{Conclusion and Further Work}
\label{more}

{\tt OOPS} seems to be a succesful proof tool for epistemic logic in  $\textbf{S5}_{(n)}$, but it would benefit from a number of improvements. First, it needs more extensive validation. We have proved neither soundness nor completion for our version of the {\bf ELtap}~\cite{these} proof system, and the set of example formulas tested is quite small.
\par Second, {\tt OOPS} can be extended in several ways which would increase its usefulness as a tool. It could be adapted to generate counterexamples, to display the proof trees belonging to semantic tableaux in a readable way, and to handle logics other than $\textbf{S5}_{(n)}$.
\par Third, a somewhat more user friendly distribution is possible. Although we have accurately documented the installation and configuration of {\tt OOPS}, it is not a straightforward process. Given that {\tt OOPS} is implemented in Java, a web-based version in the form of an applet is a possibility which could easily be exploited.

\newpage
\bibliography{mas}
\bibliographystyle{naturemag}
\appendix
\section{Propositional Algorithm}
\label{app:propalg}
\documentclass[a4paper, 10pt]{article}
\usepackage{algorithm2e}

\newcommand{\fn}[1]{{\sc #1}}

\begin{document}
\begin{algorithm}[H]
\SetLine
\KwIn{Formula, Branch, Queue}
\KwOut{Open or Closed}
Queue := \fn{clone}(Queue) \;
Branch := \fn{newBranch}(Branch) \;
\If{not Branch contains Formula}{
	\If{Formula closes Branch}{
		\Return{Closed}
	}
	\fn{add}(Branch, Formula) \;
	\fn{add}(Queue, \fn{match}(Formula)) \;
}
\While{Queue non-empty}{
	Match := \fn{removeFirst}(Queue) \;
	\eIf{Match dictates a split}{
		\For{Element $\in$ Match}{
			Result := \fn{tableau}(Element) \;
			\If{Result = Open}{
				\Return{Open}
			}
		}
		\Return{Closed}
	}{ 
		\For{Element $\in$ Match}{
			\If{not Branch contains Element}{
				\If{Element closes Branch}{
					\Return{Closed}
				}
				\fn{add}(Branch, Element) \;
				\fn{add}(Queue, Element) \;
			}
		}
	}
}
\Return{Open}
\end{algorithm}
\end{document}

\section{Modal Algorithm}
\label{app:modalg}
\documentclass[a4paper, 10pt]{article}
\usepackage{algorithm2e}

\newcommand{\fn}[1]{{\sc #1}}

\begin{document}
\begin{algorithm}[H]
\SetLine
\KwIn{Node, State}
\KwOut{Open, Closed or Error}
State = \fn{newBranchState}(State) \;

\If{not \fn{branch}(State) contains Node}{
	\If{Node closes \fn{branch}(State)}{
		\Return{Closed} \;
	}
	\fn{add}(\fn{branch}(State), Node) \;
	\fn{add}(\fn{queue}(State), \fn{match}(Node)) \;
}
\While{\fn{queue}(State) non-empty}{
	Match := \fn{removeFirst}(\fn{queue}(State)) \;
	\uIf{Match is SPLIT}{
		\For{Element $\in$ Match}{
			Result := \fn{tableau}(Element, State) \;
			\If{Result != Closed}{
				\Return{Result} \;
			}
		}
		\Return{Closed} \;
	}
	\uElseIf{Match is LINEAR}{ 
		\For{Element $\in$ Match}{
			Result := \fn{handleLinear}(Element, State) \;
			\If{Result != Open}{
				\Return{Result} \;
			}
		}
	}
	\uElseIf{Match is CREATE}{
		\For{Element $\in$ Match}{
			Result := \fn{handleCreate}(Element, State) \;
			\If{Result != Open}{
				\Return{Result} \;
			}
		}
	}
	\ElseIf{Match is ACCESS}{
		\For{Element $\in$ Match}{
			Result := \fn{handleAccess}(Element, State) \;
			\If{Result != Open}{
				\Return{Result} \;
			}
		}
	}
}
\Return{Open} \;
\caption{tableau}
\end{algorithm}

% linear case
\begin{algorithm}
\SetLine
\KwIn{Element, State}
\KwOut{Open, Closed or Error}
\If{not \fn{branch}(State) contains Element}{
	\If{Element closes \fn{branch}(State)}{
		\Return{Closed} \;
	}
	\fn{add}(\fn{branch}(State), Element) \;
	\fn{add}(\fn{queue}(State), \fn{match}(Element)) \;
}
\Return{Open} \;
\caption{handleLinear}
\end{algorithm}

% create case
\begin{algorithm}
\SetLine
\KwIn{Element, State}
\KwOut{Open, Closed or Error}

\If{\fn{creationRules}(State) entails Element}{
	\Return{Open} \;
}

\fn{add}(\fn{creationRules}(State), Element) \;
\If{Element closes \fn{branch}(State)}{
	\Return{Closed} \;
}
\fn{add}(\fn{branch}(State), Element) \;
\fn{add}(\fn{queue}(State), \fn{match}(Element)) \;
\For{ElementB $\in$ \fn{apply}(\fn{necessities}(State),
		\fn{label}(Element))}{
	\If{not \fn{branch}(State) contains ElementB}{
		\If{ElementB closes \fn{branch}(State)}{
			\Return{Closed} \;
		}
		\fn{add}(\fn{branch}(State), ElementB) \;
		\fn{add}(\fn{queue}(State), \fn{match}(ElementB)) \;
	}
}
\Return{Open} \;
\caption{handleCreate}
\end{algorithm}

% access case
\begin{algorithm}
\SetLine
\KwIn{Element, State}
\KwOut{Open, Closed or Error}

\fn{add}(\fn{necessities}(State), Element) \;
\For{ElementB $\in$ \fn{apply}(Element, \fn{labels}(\fn{branch}(State)))}{
	\If{not \fn{branch}(State) contains ElementB}{
		\If{ElementB closes \fn{branch}(State)}{
			\Return{Closed} \;
		}
		\fn{add}(\fn{branch}(State), ElementB) \;
		\fn{add}(\fn{queue}(State), \fn{match}(ElementB)) \;
	}
}
\Return{Open} \;
\caption{handleAccess}
\end{algorithm}

\end{document}

\end{document}
